\documentclass[a4]{article}
\pagestyle{myheadings}

%%%%%%%%%%%%%%%%%%%
% Packages/Macros %
%%%%%%%%%%%%%%%%%%%
\usepackage{mathrsfs}


\usepackage{fancyhdr}
\pagestyle{fancy}
\lhead{}
\chead{}
\rhead{}
\lfoot{}
\cfoot{} 
\rfoot{\normalsize\thepage}
\renewcommand{\headrulewidth}{0pt}
\renewcommand{\footrulewidth}{0pt}
\newcommand{\RomanNumeralCaps}[1]
    {\MakeUppercase{\romannumeral #1}}

\usepackage{amssymb,latexsym}  % Standard packages
\usepackage[utf8]{inputenc}
\usepackage[russian]{babel}
\usepackage{MnSymbol}
\usepackage{mathrsfs}
\usepackage{amsmath,amsthm}
\usepackage{indentfirst}
\usepackage{graphicx}%,vmargin}
\usepackage{graphicx}
\graphicspath{{pictures/}} 
\usepackage{verbatim}
\usepackage{color}
\usepackage{color,colortbl}
\usepackage[nottoc,numbib]{tocbibind}
\usepackage{float}
\usepackage{multirow}
\usepackage{hhline}

\usepackage{listings}
\definecolor{codegreen}{rgb}{0,0.6,0}
\definecolor{codegray}{rgb}{1,1,1}
\definecolor{codepurple}{rgb}{0.58,0,0.82}
\definecolor{backcolour}{rgb}{0.95,0.95,0.92}
 
\lstdefinestyle{mystyle}{
    backgroundcolor=\color{backcolour},   
    commentstyle=\color{codegreen},
    keywordstyle=\color{magenta},
    numberstyle=\tiny\color{codegray},
    stringstyle=\color{codepurple},
    basicstyle=\footnotesize,
    breakatwhitespace=false,         
    breaklines=true,                 
    captionpos=b,                    
    keepspaces=true,                 
    numbers=left,                    
    numbersep=5pt,                  
    showspaces=false,                
    showstringspaces=false,
    showtabs=false,                  
    tabsize=2
}
 
\lstset{style=mystyle}

\usepackage{url}
\urldef\myurl\url{foo%.com}
\def\UrlBreaks{\do\/\do-}
\usepackage{breakurl}
\Urlmuskip=0mu plus 1mu



\DeclareGraphicsExtensions{.pdf,.png,.jpg}% -- настройка картинок

\usepackage{epigraph} %%% to make inspirational quotes.
\usepackage[all]{xy} %for XyPic'a
\usepackage{color} 
\usepackage{amscd} %для коммутативных диграмм
%\usepackage[colorlinks,urlcolor=red]{hyperref}

%\renewcommand{\baselinestretch}{1.5}
%\sloppy
%\usepackage{listings}
%\lstset{numbers=left}
%\setmarginsrb{2cm}{1.5cm}{1cm}{1.5cm}{0pt}{0mm}{0pt}{13mm}


\newtheorem{Lemma}{Лемма}[section]
\newtheorem{Proposition}{Предложение}[section]
\newtheorem{Theorem}{Теорема}[section]
\newtheorem{Corollary}{Следствие}[section]
\newtheorem{Remark}{Замечание}[section]
\newtheorem{Definition}{Определение}[section]
\newtheorem{Designations}{Обозначение}[section]




%%%%%%%%%%%%%%%%%%%%%%% 
%Подготовка оглавления% 
%%%%%%%%%%%%%%%%%%%%%%% 
\usepackage[titles]{tocloft}
\renewcommand{\cftdotsep}{2} %частота точек
\renewcommand\cftsecleader{\cftdotfill{\cftdotsep}}
\renewcommand{\cfttoctitlefont}{\hspace{0.38\textwidth} \LARGE\bfseries} 
\renewcommand{\cftsecaftersnum}{.}
\renewcommand{\cftsubsecaftersnum}{.}
\renewcommand{\cftbeforetoctitleskip}{-1em} 
\renewcommand{\cftaftertoctitle}{\mbox{}\hfill \\ \mbox{}\hfill{\footnotesize Стр.}\vspace{-0.5em}} 
%\renewcommand{\cftchapfont}{\normalsize\bfseries \MakeUppercase{\chaptername} } 
%\renewcommand{\cftsecfont}{\hspace{1pt}} 
\renewcommand{\cftsubsecfont}{\hspace{1pt}} 
%\renewcommand{\cftbeforechapskip}{1em} 
\renewcommand{\cftparskip}{3mm} %определяет величину отступа в оглавлении
\setcounter{tocdepth}{5} 
\renewcommand{\listoffigures}{\begingroup %добавляем номер в список иллюстраций
\tocsection
\tocfile{\listfigurename}{lof}
\endgroup}
\renewcommand{\listoftables}{\begingroup %добавляем номер в список иллюстраций
\tocsection
\tocfile{\listtablename}{lot}
\endgroup}


%\renewcommand{\thelikesection}{(\roman{likesection})}
%%%%%%%%%%%
% Margins %
%%%%%%%%%%%
\addtolength{\textwidth}{0.7in}
\textheight=630pt
\addtolength{\evensidemargin}{-0.4in}
\addtolength{\oddsidemargin}{-0.4in}
\addtolength{\topmargin}{-0.4in}

%%%%%%%%%%%%%%%%%%%%%%%%%%%%%%%%%%%
%%%%%%Переопределение chapter%%%%%% 
%%%%%%%%%%%%%%%%%%%%%%%%%%%%%%%%%%%
\newcommand{\empline}{\mbox{}\newline} 
\newcommand{\likechapterheading}[1]{ 
\begin{center} 
\textbf{\MakeUppercase{#1}} 
\end{center} 
\empline} 

%%%%%%%Запиливание переопределённого chapter в оглавление%%%%%% 
\makeatletter 
\renewcommand{\@dotsep}{2} 
\newcommand{\l@likechapter}[2]{{\bfseries\@dottedtocline{0}{0pt}{0pt}{#1}{#2}}} 
\makeatother 
\newcommand{\likechapter}[1]{ 
\likechapterheading{#1} 
\addcontentsline{toc}{likechapter}{\MakeUppercase{#1}}} 




\usepackage{xcolor}
\usepackage{hyperref}
\definecolor{linkcolor}{HTML}{000000} % цвет ссылок
\definecolor{urlcolor}{HTML}{3643FF} % цвет гиперссылок
 
\hypersetup{pdfstartview=FitH,  linkcolor=linkcolor,urlcolor=urlcolor, colorlinks=true}

%%%%%%%%%%%%
% Document %
%%%%%%%%%%%%

%%%%%%%%%%%%%%%%%%%%%%%%%%%%%
%%%%%%главы -- section*%%%%%%
%%%%section -- subsection%%%%
%subsection -- subsubsection%
%%%%%%%%%%%%%%%%%%%%%%%%%%%%%
\def \newstr {\medskip \par \noindent} 



\begin{document}
\newcolumntype{g}{>{\columncolor{codegray}}c}



\def\contentsname{\LARGE{Содержание}}
\thispagestyle{empty}
\begin{center} 
\vspace{2cm} 
{\Large \sc Санкт-Петербургский Политехнический}\\
\vspace{2mm}
{\Large \sc Университет} им. {\Large\sc Петра Великого}\\
\vspace{1cm}
{\large \sc Институт прикладной математики и механики\\ 
\vspace{0.5mm}
\textsc{}}\\ 
\vspace{0.5mm}
{\large\sc Кафедра прикладной математики}\\
\vspace{15mm}
%\rule[0.5ex]{\linewidth}{2pt}\vspace*{-\baselineskip}\vspace*{3.2pt} 
%\rule[0.5ex]{\linewidth}{1pt}\\[\baselineskip] 
{\huge \sc Лабораторная работа №$7$\\
	Проверка гипотезы о законе распределения генеральной совокупности. 
	Метод хи-квадрат.
	\vspace{6mm}
	
}
\vspace*{2mm}
%\rule[0.7ex]{\linewidth}{1pt}\vspace*{-\baselineskip}\vspace{3.2pt} 
%\rule[0.5ex]{\linewidth}{2pt}\\ 
\vspace{6cm} 
Студент группы $3630102/70301$ \hfill Камянский Д.В.\\
\vspace{1cm}
Преподаватель \hfill Баженов А. Н.\\
\vspace{20mm} 


\vfill {\large\textsc{Санкт-Петербург}}\\ 
2020 г.
\end{center}

%%%%%%%%%%%%%%%%%%%%%%%%%%%%%%%%%%%%%%%%%%%%%%%%%%%%%%%%%%%%%%%%%%%%%%%%%%%%%%%%%%%%%%%%%%%%%%
%\ \\[4cm]

%\rm
%%%%%%%%%%%%%%%%%%%%%%%%%%%%%%%%%%%%%%%%%%%%%%%%%%%%%%%%%%%%%%%%%%%%%%%%%%%%%%%%%%%%%%%%%%%%%%
\newpage
\pagestyle{plain}

%\begin{center}
%\begin{abstract} 

%\end{abstract}

%\end{center}

\newpage
\tableofcontents{}
\newpage
\listoftables{}
\newpage

\section{Постановка задачи}

Необходимо сгенерировать выборку объемом $100$ элементов для нормального распределения $N(x;0,1).$ По сгенерированной выборке оценить параметры $\mu$ и $\sigma$ нормального закона методом максимального правдоподобия. В качестве основной гипотезы $H_0$ будем считать, что сгенерированное распределение имеет вид $N(x,\overset{\wedge}{\mu},\overset{\wedge}{\sigma} ).$ Проверить основную гипотезу, используя критерий согласия $\chi$. В качестве ровня значимости взять $\alpha=0,05.$ Привести таблицу вычислений $\chi^2.$

\section{Теория}
\subsection{Метод максимального правдоподобия}
Метод максимального правдоподобия $\--$ метод оценивания неизвестного параметра путём максимимзации функции правдоподобия.
\begin{equation}
    \overset{\wedge}{\theta}_{\text{МП}}=argmax \mathbf{L}(x_1,x_2,\ldots,x_n,\theta)
\end{equation}

Где $\mathbf{L}$ это функция правдоподобия, которая представляет собой совместную плотность вероятности независимых случайных величин $X_1,x_2,\ldots,x_n$ и является функцией неизвестного параметра $\theta$
\begin{equation}
    \mathbf{L} = f(x_1,\theta)\cdot f(x_2,\theta)\cdot\cdots\cdot f(x_n,\theta)
\end{equation}
Оценкой максимального правдоподобия будем называть такое значение $\overset{\wedge}{\theta}_{\text{МП}}$ из множества допустимых значений параметра $\theta,$ для которого функция правдоподобия принимает максимальное значение при заданных $x_1,x_2,\ldots,x_n.$

Тогда при оценивании математического ожидания $m$ и дисперсии $\sigma^2$ нормального распределения $N(m,\sigma)$ получим:
\begin{equation}
    \ln(\mathbf{L})=-\frac{n}{2}\ln(2\pi)-\frac{n}{2}\ln\left(\sigma^2\right)-\frac{1}{2\sigma^2}\sum\limits_{i=1}^n(x_i-m)^2
\end{equation}

Отсюда находятся выражения для оценок $m$ и $\sigma^2$:
\begin{equation}
\begin{cases}
&  m= \bar{x}\\ 
&  \sigma^{2} = s^{2}
\end{cases}
\end{equation}

\subsection{Критерий согласия Пирсона}
Разобьём генеральную совокупность на $k$ неперсекающихся подмножеств $\Delta_1, \Delta_2,\ldots, \Delta_k,\;\Delta_i = (a_i,a_{i+1}],$ $p_i = P(X\in\Delta_i),\;i=1,2,\ldots,k\; \--$ вероятность того, что точка попала в $i$ый промежуток.

Так как генеральная совокупность это $\mathbb{R},$ то крайние промежутки будут бесконечными: $\Delta_1=(-\infty,a_1],\;\Delta_k=(a_k,\infty),\;p_i = F(a_i)-F(a_{i-1})$

$n_i\;\--$ частота попадания выборочных элементов в $\Delta_i,\;i=1,2,\ldots,k.$

В случае справедливости гипотезы $H_0$ относительно частоты $\frac{n_i}{n}$ при больших $n$ должны быть близки к $p_i,$ значит в качестве меры имеет смысл взять: 
\begin{equation}
    Z = \sum\limits_{i=1}^k\frac{n}{p_i}\left(\frac{n_i}{n}-p_i\right)^2
\end{equation}
Тогда
\begin{equation}
    \chi^2_B=\sum\limits_{i=1}^k\frac{n}{p_i}\left(\frac{n_i}{n}-p_i\right)^2=\sum\limits_{i=1}^k\frac{(n_i-np_i)^2}{np_i}
\end{equation}
Для выполнения гипотезы $H_0$ должны выполняться следующие условия:
\begin{equation}
    \chi_B^2 < \chi_{1-\alpha}^2(k-1)
\end{equation}
где $\chi_{1-\alpha}^2(k-1)\;\--$ квантиль распределения $\chi^2$ с $k-1$ степенями свободы порядка $1-\alpha,$ где $\alpha$ заданный уровень значимости.
\section{Реализация}
Работы была выполнена на языке $Python 3.8.2$
Для генерации выборок использовался модуль .
Для генерации выборок и обработки функции распределения использовалась библиотека $scipy.stats$.

\section{Результаты}
\subsection{Метод максимального правдоподобия}

При подсчете оценок параметров закона нормального распределения методом максимального правдоподобия были получены следующие значения:
\begin{equation}
\begin{split}
    &\overset{\wedge}{m}_{\text{МП}} = 0.090527\\
   &  \overset{\wedge}{\sigma}^2_{\text{МП}} = 0.963167
\end{split}
\end{equation}
\subsection{Критерий Пирсона}
\begin{table}[H]
\caption{Таблица вычислений $\chi^2$}
\label{tab:my_label1}
\begin{center}
\vspace{5mm}
\begin{tabular}{|c|c|c|c|c|}
\hline
 i & $\Delta_i$ & $n_i$ & $p_i$ & $\frac{(n_i-np_i)^2}{np_i}$\\
\hline
1&	 $(-\infty, -1.0]$ &	15  &	 0.1288 &	 0.3501\\
\hline
2&	$(-1.0, -0.5)$&	10&	 0.1411&	  1.1988\\
\hline
3& $(-0.5, 0.0)$&	24&	 0.1927&	 1.1634\\
\hline
4&	$(0.0, 0.5)$&	19&	 0.2021&	 0.0721\\
\hline
5& $(0.5, 1.0)$&	13&	 0.1629&	 0.6626\\
\hline
6& $(1.0, \infty)$&	19&	 0.1725&	 0.1771\\
\hline
$\sum$&&		100&	1  & 3.6241	\\

\hline
\end{tabular}
\end{center}
\end{table}

$$\chi_B^2= 3.6241$$

\subsection{Проверка гипотезы о нормальности для распределения \newline  Лапласа}
Размер выборки $ n = 25 $ для распределения Лапласа
\begin{equation}\label{eqn:laplace}
L\left( x,0,\frac{1}{\sqrt{2}}\right) = \frac{1}{\sqrt{2}}e^{-\sqrt{2}\vert x\vert}
\end{equation}
  \begin{equation}
  \begin{split}
  &\overset{\wedge}{m}_{\text{МП}} = 0.198045\\
  &  \overset{\wedge}{\sigma}^2_{\text{МП}} = 0.656187
  \end{split}
  \end{equation}

\begin{table}[H]
	\caption{Таблица вычислений $\chi^2$}
	\label{tab:my_label1}
	\begin{center}
		\vspace{5mm}
		\begin{tabular}{|c|c|c|c|c|}
			\hline
			i & $\Delta_i$ & $n_i$ & $p_i$ & $\frac{(n_i-np_i)^2}{np_i}$\\
			\hline
			1&	 $(-\infty, -1.0]$ &	1  &	 0.0339 &	 0.027\\
			\hline
			2&	$(-1.0, 0.0)$&	12&	 0.3475&	  1.2641\\
			\hline
			3& $(0.0, 1.0)$&	8&	 0.5078&	 1.7359\\
			\hline
			4&	$(1.0, \infty)$&	4&	 0.1108&	 0.5454\\
			\hline
			
			\hline
		\end{tabular}
	\end{center}
\end{table}

$$\chi_B^2 = 3.5725$$ 

\section{Выводы}

Табличное значение квантиля  $\chi^2_{1-\alpha}(k-1)=\chi^2_{0.95}(5) = 11.0705$.
Полученное значение критерия согласия Пирсона для нормального распределения $\chi_B^2 = 3.6241 < \chi^2_{0.95}(5)$, следовательно основная гипотеза $H_0$ на исходной выборке не может быть отвергнута на уровне значимости $\alpha = 0.05.$. Для распределения Лапласа полученное значение критерия Пирсона $\chi_B^2 = 3.5725 < \chi^2_{0.95}(3) = 7.8147$ означает что из полученной выборки мы не можем отвергнуть гипотезу $H_0$ о нормальности исходного распределения. Такой результат легко объясним низким размером выборки, так как интервалы в которых мы оцениваем распределение получаются слишком большими, на которых распределение Лапласа очень схоже с нормальным.


\begin{thebibliography}{}
   \bibitem{numpy}  Модуль numpy  -  \url{https://physics.susu.ru/vorontsov/language/numpy.html}
    
   \bibitem{skp}
   Модуль scipy - \url{https://docs.scipy.org/doc/scipy/reference/}


\bibitem{chi_quant}
Таблица значений $\chi^2$ - \url{http://statsoft.ru/home/textbook/modules/sttable.html#chi}

\end{thebibliography}

\section{Приложения}


Код лаборатрной:\; \url{https://github.com/dkamianskii/MatStatLabs/tree/master/Lab7}

%\lstinputlisting[language=Python]{MatStatLab7.py}

\end{document}
