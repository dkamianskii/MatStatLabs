\documentclass[a4]{article}
\pagestyle{myheadings}

%%%%%%%%%%%%%%%%%%%
% Packages/Macros %
%%%%%%%%%%%%%%%%%%%
\usepackage{mathrsfs}


\usepackage{fancyhdr}
\pagestyle{fancy}
\lhead{}
\chead{}
\rhead{}
\lfoot{}
\cfoot{} 
\rfoot{\normalsize\thepage}
\renewcommand{\headrulewidth}{0pt}
\renewcommand{\footrulewidth}{0pt}
\newcommand{\RomanNumeralCaps}[1]
    {\MakeUppercase{\romannumeral #1}}

\usepackage{amssymb,latexsym}  % Standard packages
\usepackage[utf8]{inputenc}
\usepackage[russian]{babel}
\usepackage{MnSymbol}
\usepackage{mathrsfs}
\usepackage{amsmath,amsthm}
\usepackage{indentfirst}
\usepackage{graphicx}%,vmargin}
\usepackage{graphicx}
\graphicspath{{pictures/}} 
\usepackage{verbatim}
\usepackage{color}
\usepackage[nottoc,numbib]{tocbibind}
\usepackage{float}
\usepackage{enumitem}

\usepackage{listings}
\definecolor{codegreen}{rgb}{0,0.6,0}
\definecolor{codegray}{rgb}{0.5,0.5,0.5}
\definecolor{codepurple}{rgb}{0.58,0,0.82}
\definecolor{backcolour}{rgb}{0.95,0.95,0.92}
 
\lstdefinestyle{mystyle}{
    backgroundcolor=\color{backcolour},   
    commentstyle=\color{codegreen},
    keywordstyle=\color{magenta},
    numberstyle=\tiny\color{codegray},
    stringstyle=\color{codepurple},
    basicstyle=\footnotesize,
    breakatwhitespace=false,         
    breaklines=true,                 
    captionpos=b,                    
    keepspaces=true,                 
    numbers=left,                    
    numbersep=5pt,                  
    showspaces=false,                
    showstringspaces=false,
    showtabs=false,                  
    tabsize=2
}
 
\lstset{style=mystyle}

\usepackage{url}
\urldef\myurl\url{foo%.com}





\DeclareGraphicsExtensions{.pdf,.png,.jpg}% -- настройка картинок

\usepackage{epigraph} %%% to make inspirational quotes.
\usepackage[all]{xy} %for XyPic'a
\usepackage{color} 
\usepackage{amscd} %для коммутативных диграмм
%\usepackage[colorlinks,urlcolor=red]{hyperref}

%\renewcommand{\baselinestretch}{1.5}
%\sloppy
%\usepackage{listings}
%\lstset{numbers=left}
%\setmarginsrb{2cm}{1.5cm}{1cm}{1.5cm}{0pt}{0mm}{0pt}{13mm}


\newtheorem{Lemma}{Лемма}[section]
\newtheorem{Proposition}{Предложение}[section]
\newtheorem{Theorem}{Теорема}[section]
\newtheorem{Corollary}{Следствие}[section]
\newtheorem{Remark}{Замечание}[section]
\newtheorem{Definition}{Определение}[section]
\newtheorem{Designations}{Обозначение}[section]




%%%%%%%%%%%%%%%%%%%%%%% 
%Подготовка оглавления% 
%%%%%%%%%%%%%%%%%%%%%%% 
\usepackage[titles]{tocloft}
\renewcommand{\cftdotsep}{2} %частота точек
\renewcommand\cftsecleader{\cftdotfill{\cftdotsep}}
\renewcommand{\cfttoctitlefont}{\hspace{0.38\textwidth} \LARGE\bfseries} 
\renewcommand{\cftsecaftersnum}{.}
\renewcommand{\cftsubsecaftersnum}{.}
\renewcommand{\cftbeforetoctitleskip}{-1em} 
\renewcommand{\cftaftertoctitle}{\mbox{}\hfill \\ \mbox{}\hfill{\footnotesize Стр.}\vspace{-0.5em}} 
%\renewcommand{\cftchapfont}{\normalsize\bfseries \MakeUppercase{\chaptername} } 
%\renewcommand{\cftsecfont}{\hspace{1pt}} 
\renewcommand{\cftsubsecfont}{\hspace{1pt}} 
%\renewcommand{\cftbeforechapskip}{1em} 
\renewcommand{\cftparskip}{3mm} %определяет величину отступа в оглавлении
\setcounter{tocdepth}{5} 
\renewcommand{\listoffigures}{\begingroup %добавляем номер в список иллюстраций
\tocsection
\tocfile{\listfigurename}{lof}
\endgroup}
\renewcommand{\listoftables}{\begingroup %добавляем номер в список иллюстраций
\tocsection
\tocfile{\listtablename}{lot}
\endgroup}


   
   
%\renewcommand{\thelikesection}{(\roman{likesection})}
%%%%%%%%%%%
% Margins %
%%%%%%%%%%%
\addtolength{\textwidth}{0.7in}
\textheight=630pt
\addtolength{\evensidemargin}{-0.4in}
\addtolength{\oddsidemargin}{-0.4in}
\addtolength{\topmargin}{-0.4in}

%%%%%%%%%%%%%%%%%%%%%%%%%%%%%%%%%%%
%%%%%%Переопределение chapter%%%%%% 
%%%%%%%%%%%%%%%%%%%%%%%%%%%%%%%%%%%
\newcommand{\empline}{\mbox{}\newline} 
\newcommand{\likechapterheading}[1]{ 
\begin{center} 
\textbf{\MakeUppercase{#1}} 
\end{center} 
\empline} 

%%%%%%%Запиливание переопределённого chapter в оглавление%%%%%% 
\makeatletter 
\renewcommand{\@dotsep}{2} 
\newcommand{\l@likechapter}[2]{{\bfseries\@dottedtocline{0}{0pt}{0pt}{#1}{#2}}} 
\makeatother 
\newcommand{\likechapter}[1]{ 
\likechapterheading{#1} 
\addcontentsline{toc}{likechapter}{\MakeUppercase{#1}}} 




\usepackage{xcolor}
\usepackage{hyperref}
\definecolor{linkcolor}{HTML}{000000} % цвет ссылок
\definecolor{urlcolor}{HTML}{3643FF} % цвет гиперссылок
 
\hypersetup{pdfstartview=FitH,  linkcolor=linkcolor,urlcolor=urlcolor, colorlinks=true}

%%%%%%%%%%%%
% Document %
%%%%%%%%%%%%

%%%%%%%%%%%%%%%%%%%%%%%%%%%%%
%%%%%%главы -- section*%%%%%%
%%%%section -- subsection%%%%
%subsection -- subsubsection%
%%%%%%%%%%%%%%%%%%%%%%%%%%%%%
\def \newstr {\medskip \par \noindent} 



\begin{document}
\def\contentsname{\LARGE{Содержание}}
\thispagestyle{empty}
\begin{center} 
\vspace{2cm} 
{\Large \sc Санкт-Петербургский Политехнический}\\
\vspace{2mm}
{\Large \sc Университет} им. {\Large\sc Петра Великого}\\
\vspace{1cm}
{\large \sc Институт прикладной математики и механики\\ 
\vspace{0.5mm}
\textsc{}}\\ 
\vspace{0.5mm}
{\large\sc Кафедра прикладной математики}\\
\vspace{15mm}
%\rule[0.5ex]{\linewidth}{2pt}\vspace*{-\baselineskip}\vspace*{3.2pt} 
%\rule[0.5ex]{\linewidth}{1pt}\\[\baselineskip] 
{\huge \sc Лабораторная работа №$2$\\
	Характеристики положения выборки
	\vspace{6mm}
	
}
\vspace*{2mm}
%\rule[0.7ex]{\linewidth}{1pt}\vspace*{-\baselineskip}\vspace{3.2pt} 
%\rule[0.5ex]{\linewidth}{2pt}\\ 
\vspace{6cm} 
Студент группы $3630102/70301$ \hfill Камянский Д.В.\\
\vspace{1cm}
Преподаватель \hfill Баженов А. Н.\\
\vspace{20mm} 


\vfill {\large\textsc{Санкт-Петербург}}\\ 
2020 г.
\end{center}

%%%%%%%%%%%%%%%%%%%%%%%%%%%%%%%%%%%%%%%%%%%%%%%%%%%%%%%%%%%%%%%%%%%%%%%%%%%%%%%%%%%%%%%%%%%%%%
%\ \\[4cm]

%\rm
%%%%%%%%%%%%%%%%%%%%%%%%%%%%%%%%%%%%%%%%%%%%%%%%%%%%%%%%%%%%%%%%%%%%%%%%%%%%%%%%%%%%%%%%%%%%%%
\newpage
\pagestyle{plain}

%\begin{center}
%\begin{abstract} 

%\end{abstract}

%\end{center}

\newpage
\tableofcontents{}
\newpage
\listoftables{}
\newpage

\section{Постановка задачи}

Любыми средствами сгенерировать выборки размеров $10,$ $100,$ $1000$ элементов для $5$ти распределений. Для каждой выборки вычислить $\overline{x},\; med\; x,\; Z_R,\; Z_Q,\; Z_{tr},$ при $r = \frac{n}{4}.$

Распределения:
\begin{equation}\label{eqn:normal}
N(x,0,1) = \frac{1}{\sqrt{2\pi}}e^{-\frac{x^2}{2}}
\end{equation} 

\begin{equation}\label{eqn:cauchy}
C(x,0,1) = \frac{1}{\pi(1+x^2)}
\end{equation}

\begin{equation}\label{eqn:laplace}
L\left( x,0,\frac{1}{\sqrt{2}}\right) = \frac{1}{\sqrt{2}}e^{-\sqrt{2}\vert x\vert}
\end{equation}

\begin{equation}\label{eqn:poisson}
P(5,k) = \frac{5^k}{k!}e^{-5}
\end{equation}  

\begin{equation}\label{eqn:uniform}
M(x,-\sqrt{3}, \sqrt{3}) = 
\begin{cases}
\frac{1}{2\sqrt{3}} &\vert x\vert \leqslant \sqrt{3}\\
0 &\vert x\vert > \sqrt{3}
\end{cases}
\end{equation}
\section{Теория}

\begin{enumerate}
\item Выборочное среднее:
\begin{equation}\label{eqn:average}
\overline{x} = \frac{1}{n}\sum_{i=1}^n x_i \hfill  
\end{equation}
\item Выборочная медиана:
\begin{equation}
med\; x = \begin{cases}
x_{k+1}, & n = 2k+1\\
\frac{1}{2}\left(x_k+x_{k+1}\right), & n = 2k
\end{cases} \hfill  \label{eqn:med}
\end{equation}
\item Полусумма экстремальных значений:
\begin{equation}
Z_R = \frac{1}{2}\left(x_1+x_n\right) \hfill  \label{eqn:mean_extr}
\end{equation}
\item Полусумма квартилей:
\begin{equation}
Z_Q = \frac{1}{2}\left(Z_{\frac{1}{4}}+Z_{\frac{3}{4}}\right) \hfill  \label{eqn:quartiles}
\end{equation}
\item Усечённое среднее:
\begin{equation}
Z_{tr} = \frac{1}{n - 2r}\sum_{i=r+1}^{n-r} x_i \hfill  \label{eqn:cut_mean}
\end{equation}
\end{enumerate}
\vspace{2mm}
После вычисления характеристик положения $1000$ раз находится среднее значение и дисперсия: 
\begin{equation}
E(z) = \frac{1}{n}\sum_{i=1}^n z_i
\end{equation} 
\begin{equation}
D(z) = E\left(z^2\right) - E^2(z)
\end{equation}

\section{Реализация}
Для генерации выборки был использован $Python\;3.8.2$: модуль $stats$ библиотеки $scipy$ для генерации выборок различных распределений.Характеристики положения были вычислены с помощью библиотеки $numpy$.

\newpage
\section{Результаты}

\begin{table}[H]
\caption{\label{tab:normal} Стандартное нормальное распределение.}
\begin{center}
\begin{tabular}{|c|c|c|c|c|c|}
\hline
$n = 10$ & average & med & $Z_R$ & $Z_Q$ & $Z_{tr}$\\
\hline
$E =$ & $-0.0$ & $-0.02$ & $0.0$ & $-0.0$ & $-0.0$ \\
\hline
$D =$ & $0.106931$ & $0.145363$ & $0.186191$ & $0.111027$ & $0.165807$ \\
\hline
$n = 100$ & average & med & $Z_R$ & $Z_Q$ & $Z_{tr}$\\
\hline
$E =$ & $-0.0$ & $0.01$ & $0.02$ & $0.0$ & $0.0$\\
\hline
$D =$ & $0.010365$ &  $0.015172$ &  $0.092881$ &  $0.011974$ &  $0.017426$\\
\hline
$n = 1000$ & average & med & $Z_R$ & $Z_Q$ & $Z_{tr}$\\
\hline
$E =$ & $0.0$ & $0.0$ & $-0.0$ & $0.0$ & $0.0$\\
\hline
$D =$ & $0.00096$ &  $0.001594$ &  $0.060543$ &  $0.001256$ &  $0.001888$\\
\hline
\end{tabular}
\end{center}
\end{table}

\begin{table}[H]
\caption{\label{tab:cauchy} Стандартное распределение Лапласа.}
\begin{center}
\begin{tabular}{|c|c|c|c|c|c|}
\hline
$n = 10$   & average & med & $Z_R$ & $Z_Q$ & $Z_{tr}$\\ \hline
$E =$ & $-0.01$ & $0.01$ &  $-0.0$ &  $-0.0$ &  $-0.01$\\ \hline
$D =$  & $0.097868$ &  $0.069859$ &  $0.390201$ &  $0.093469$ &  $0.167235 $\\    \hline
					
$n = 100$   & average & med & $Z_R$ & $Z_Q$ & $Z_{tr}$\\ \hline
$E =$ &	$-0.0$ &  $0.01$ &  $0.01$ &  $0.0$ &  $-0.0$\\   \hline
$D =$  & $0.009778$ &  $0.005878$ &  $0.429639$ &  $0.009072$ &  $0.020247$\\   \hline 
					
$n = 1000$   & average & med & $Z_R$ & $Z_Q$ & $Z_{tr}$\\ \hline
$E =$ & $0.0$ &  $0.0$ &  $-0.02$ &  $-0.0$ &  $0.0$\\  \hline
$D =$ & $0.001009$ &  $0.000531$ &  $0.428975$ &  $0.000997$ &  $0.001992$\\    
\hline
\end{tabular}
\end{center}
\end{table}

\begin{table}[H]
\caption{\label{tab:laplace} Распределение Коши.}
\begin{center}
\begin{tabular}{|c|c|c|c|c|c|}
\hline
$n = 10$    & average & med & $Z_R$ & $Z_Q$ & $Z_{tr}$\\ \hline 
$E = $  & $0.31$ &  $-0.01$ &  $-5.87$ &  $0.03$ &  $3.5$\\ \hline
$D = $  & $260.978036$ &  $0.347366$ &  $26462.674037$ &  $0.774488$ &  $6529.2454$\\ \hline
					
$n = 100$  & average & med & $Z_R$ & $Z_Q$ & $Z_{tr}$\\ \hline
$E = $ & $2.26$ &  $0.01$ &  $22.79$ &  $-0.0$ &  $1.71$   \\ \hline
$D =$ & $2122.106445$ &  $0.023531$ &  $1741103.211025$ &  $0.05289$ &  $1778.363557$    \\ \hline
					
$n = 1000$   & average & med & $Z_R$ & $Z_Q$ & $Z_{tr}$\\ \hline
$E =$ & $1.14$ &  $0.0$ &  $-1845.52$ &  $-0.0$ &  $-1.64$   \\ \hline
$D = $ & $706.431432$ &  $0.002497$ &  $21117320297.542297$ &  $0.004864$ &  $1675.403679$    \\ 
\hline
\end{tabular}
\end{center}
\end{table}

\begin{table}[H]
\caption{\label{tab:uniform} Равномерное распределение.}
\begin{center}
\begin{tabular}{|c|c|c|c|c|c|}
\hline
$n = 10$  & average & med & $Z_R$ & $Z_Q$ & $Z_{tr}$\\ \hline
$E =$ &	$-0.02$  & 	$0.0$  & 	$0.0$    &	$-0.0$   &	$-0.02$  \\ \hline  
$D =$ &	$0.095612$  &  $0.227685$  &  $0.043735$  &  $0.135551$  &  $0.171826$    \\ \hline
					
$n = 100$  & average & med & $Z_R$ & $Z_Q$ & $Z_{tr}$\\ \hline
$E =$  &	$0.0$   & 	$-0.0$   &	$-0.0$   &	$0.0$   & 	$0.01$    \\ \hline
$D =$ &	$0.009971$   &  $0.029194$   &  $0.00058$   &  $0.014725$   &  $0.019723$  \\ \hline
					
$n = 1000$  & average & med & $Z_R$ & $Z_Q$ & $Z_{tr}$\\ \hline
$E =$  &  	$-0.0$    &	$0.0$    &	$-0.0 $  & 	$0.0$   & 	$-0.0$    \\ \hline
$D =$ & $0.000959$   &  $0.002864$   &  $6e-06$   &  $0.001502$   &  $0.00205$    \\
\hline
\end{tabular}
\end{center}
\end{table}

\begin{table}[H]
\caption{\label{tab:poisson} Распределение Пуассона.}
\begin{center}
\begin{tabular}{|c|c|c|c|c|c|}
\hline
$n = 10$   & average & med & $Z_R$ & $Z_Q$ & $Z_{tr}$\\ \hline
$E =$ & $5.01$ &  $4.84$ &  $5.3$ &  $4.88$ &  $5.02$    \\ \hline
$D =$ & $0.515302$ &  $0.739579$ &  $0.942244$ &  $0.586539$ &  $0.832244$    \\ \hline
					
$n = 100$   & average & med & $Z_R$ & $Z_Q$ & $Z_{tr}$\\ \hline
$E =$ & $5.0$ &  $4.9$ &  $5.97$ &  $4.89$ &  $4.99$ \\ \hline
$D =$ &	$0.049874$ &  $0.102596$ &  $0.499128$ &  $0.143341$ &  $0.101492$ \\ \hline
					
$n = 1000$   & average & med & $Z_R$ & $Z_Q$ & $Z_{tr}$\\ \hline
$E =$ & $5.0$ &  $5.0$ &  $6.83$ &  $4.67$ &  $5.0$    \\ \hline
$D =$ & $0.005329$ &  $0.0$ &  $0.341759$ &  $0.076798$ &  $0.010249$    \\
\hline
\end{tabular}
\end{center}
\end{table}


\section{Обсуждение}
\par При вычислении средних значений приходится отбрасывать некоторое число знаков после запятой, так как получаемая дисперсия не может гарантировать получаемое точное значение. \par Другими словами дисперсия может гарантировать порядок точности среднего значения только до первого значащего знака после запятой в дисперсии включительно. \par Исключением является стандартное распределение Коши, так как оно имеет бесконечную дисперсию, а значит не может гарантировать никакой точности, а в общем мат. ожидание для данного распределения не определено.

\section{Выводы}

\par В процессе работы вычислены значения характеристик положения для определённых распределений на выборках фиксированной мощности и получено следующее ранжирование характеристик положения:

\begin{enumerate}
    \item Стандартное нормальное распределение $$\overline{x} < Z_{tr} < Z_Q < med\;x < Z_R$$
    
    \item Распределение Лапласа (коэффициент масштаба $\sqrt{2}$ коэффициент сдвига равен нулю) $$Z_Q < Z_{tr} < \overline{x} <  Z_R <med\;x$$
    
    \item Стандартное распределение Коши $$med\;x < Z_Q <  \overline{x} < Z_{tr} < Z_R$$
    
    \item Равномерное распределение на отрезке $\left[-\sqrt{3},\sqrt{3}\right]$ $$Z_R < med\;x < Z_Q < Z_{tr} < \overline{x}$$
    
    \item Распределение Пуассона (значение мат ожидания равно $5$) $$Z_Q < med\;x < \overline{x} < Z_{tr} < Z_R$$
    
\end{enumerate}



\begin{thebibliography}{}
    \href{https://docs.scipy.org/doc/scipy/reference/stats.html}{Модуль scipy.stats}\\
    \href{https://docs.scipy.org/doc/numpy/reference/routines.statistics.html}{Модуль numpy для статистики}
\end{thebibliography}

\section{Приложения}


Код лаборатрной:\; \url{https://github.com/dkamianskii/MatStatLabs/tree/master/Lab2}

%\lstinputlisting[language=Python]{MatStatLab2.py}

\end{document}
